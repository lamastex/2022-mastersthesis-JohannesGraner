% !TEX root = $uni/master-thesis/thesis/main.tex
\documentclass[../main.tex]{subfiles}

\begin{document}

  When handling multi-dimensional data in a computer, 
  there is a problem of how to represent the data in an efficient manner.
  A common family of data structures for this is trees.
  This family includes, among many others, binary search trees, quadtrees, and k-d trees.
  It is often the case that a specific type of tree is 
  developed to suit a particular data type \cite{mrp-raaz-harlow-tucker}.
  Mapped regular paving is a type of tree that is well suited for 
  representing multi-dimensional data that is piece-wise constant, or 
  can be well approximated by piece-wise constant functions.
  Moreover, arithmetic operation can be performed over mapped regular pavings.

  A notable use-case is density estimation through histograms, 
  where the density $f$ for any continuous random variable can be 
  consistently estimated given some conditions on the growth of the histogram 
  \cite{histogram-consistency-nobel-lugosi},
  with universal performance guarantees for a given data sample of size $n$ \cite{devroye-lugosi}.

  \section{Regular pavings (RPs)}
    \subfile{mapped-regular-pavings/regular-pavings.tex}

  \section{Mapped regular pavings (MRPs)}
    \subfile{mapped-regular-pavings/mrp.tex}

  \section{MRPs as histograms}
    \subfile{mapped-regular-pavings/histograms.tex}

  \section{Collated regular pavings (CRPs)}
    \subfile{mapped-regular-pavings/crp.tex}

\end{document}