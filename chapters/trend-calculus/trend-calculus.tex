% !TEX root = $uni/master-thesis/thesis/main.tex
\documentclass[../trend-calculus.tex]{subfiles}

\begin{document}
  Time series can fluctuate rapidly, and need to be smoothed before determining trends.
  The smoothing is accomplished by only considering the 
  highest and lowest point within a window of specified size.
  By increasing the window size, smaller fluctuations inside the window are disregarded as noise, 
  while the highest and lowest points are taken to be indicative of the current trend.

  In the Trend Calculus algorithm, an up trend is defined as a period with 
  ``higher highs and higher lows", while a down trend has ``lower highs and lower lows".

 A complete description of the Trend Calculus algorithm is as follows.

  \subsection{Algorithm}
    \subfile{trend-calculus/algorithm.tex}

  \subsection{Trend Calculus as a sketch}
    \subfile{trend-calculus/sketch.tex}

  \subsection{Implementation}
    \subfile{trend-calculus/spark.tex}

  \subsection{Trinomial Random Walk Showcase}
    \subfile{trend-calculus/random-walk.tex}
\end{document}
