% !TEX root = $uni/master-thesis/thesis/main.tex
\documentclass[../trend-calculus.tex]{subfiles}

\begin{document}
  The Trend Calculus algorithm is easily applicable to 
  streaming time series where each point can only be read once, and 
  memory usage while processing is not allowed to grow with the input size $N$ 
  (or at least grow very slowly compared to $N$).
  Note that almost every step in the algorithm 
  depends only on the current and last window.
  The only exception is if all orders of reversals are to be computed in a single pass, 
  in which case the last window for each reversal order must be kept in memory, 
  as well as a buffer of reversals of lower order that will make up the next window.

  A noteworthy feature of the algorithm when applied 
  to a streaming time series is that the order of 
  previously identified reversal points may increase over time.
  This is a consequence of needing two complete windows ($2s$ points)
  of reversal points of a given order before a higher order reversal can be identified.

  A single window consists of two sets (high and low) of two floating point values 
  (time and value), and is therefore very cheap to store.
  In the worst case scenario,
  i.e. $s = 2$ and each window contains a reversal,
  the number of reversal orders grows linearly with $N$, but
  in practice the number of reversal orders grow logarithmically with $N$ \cite{trendcalculus}.
  
  Hence, $\mathcal{O}(\log N)$ need to be stored to compute all reversal orders.
  This grows slowly enough that it is suitable for streaming applications.
  As an example, a financial time series with minute-level resolution over 10 years 
  ($N \approx 2.86 \cdot 10^6$) was shown to have 15 reversal orders \cite{spark-trendcalculus-examples}.
\end{document}