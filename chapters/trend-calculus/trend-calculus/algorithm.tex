% !TEX root = $uni/master-thesis/thesis/main.tex
\documentclass[../trend-calculus.tex]{subfiles}

\begin{document}
  The input to the algorithm is a window size $s \ge 2$ and a time series $T$ with $N \ge 2s$ observations.

  \begin{equation*}
    T = \{ p_i = (t_i, x_i) \in \mathbb{R}^2 : t_i < t_j, 1 \le i < j \le N \}
  \end{equation*}
  To create the windows, define the function $f_s : T^s \to T^2$ as

  \begin{align*}
    f_s(p_1, \dots, p_s) = ( & \max_{t_i} (\min_{x_i} \{ (t_i, x_i) : i = 1,\dots,s \}), \\
    & \min_{t_i} (\max_{x_i} \{ (t_i, x_i) : i = 1,\dots,s \}) )
  \end{align*}
  which finds the latest minimum and earliest maximum among the points $p_1,\dots,p_s$. 

  Assume that $N$ is divisible by $s$ 
  (otherwise, only consider $N - (N \mod s)$ points of the series).
  Let $n = \frac{N}{s} \in \Zz_{\ge 2}$ and 

  \begin{equation*}
    W = \{ w_k = (l_k, h_k) = f_s(p_{1+s(k-1)},\dots,p_{s+s(k-1)}) : k = 1,\dots,n\}. 
  \end{equation*}
  Then $W$ is the set of $n$ non-overlapping windows of size $s$ that summarize the time series $T$ by 
  the minimum (low) $l_k$ and maximum (high) $h_k$.
  For each $w_k \in W$, define the trend for that window as 

  \begin{equation*}
    \tau_k = \begin{cases}
      0, & k = 1 \\
      \tau(w_{k-1}, w_k), & k > 1
    \end{cases}
  \end{equation*}
  where $\tau : (\Rz^2)^2 \to \{-1,0,1\}$ is given by

  \begin{equation*}
    \tau(w_{k-1}, w_k) = \sign(\sign \circ \pi_x(l_k - l_{k-1}) + \sign \circ \pi_x(h_k - h_{k-1}))
  \end{equation*}
  where $\pi_x : \mathbb{R}^2 \to \mathbb{R}$ is the projection $(t,x) \mapsto x$.

  Once an initial trend has been identified, 
  each point thereafter is either part of an up or a down trend.
  However, there may be points after the initial period that are assigned a zero trend.
  The following step corrects this.
  Let $\tilde n = \min \{ k : \tau_k \le 0 \}$ be the index of the first window to have a non-zero trend.
  Note that $\tilde n \ge 2$ since $\tau_1 = 0$ by definition, and let

  \begin{equation*}
    w_k^* = \begin{cases}
      w_k, & k \le \tilde n \\
      w_{k - \#\{\tilde n < j < k : t_j = 0\}}, & \tau_k \ne 0 \text{ and } k > \tilde n \\
      (\min_x\{a,b\}, \max_x\{a,b\}), & \tau_k = 0 \text{ and } k > \tilde n 
    \end{cases}
  \end{equation*}
  where $a = \max_t\{l_{k-1}, h_{k-1}\}, b = \min_t\{l_k, h_k\}$.
  This creates intermediate windows at every point where 
  the trend is 0 after the first trend has been identified, 
  and inserts them at the correct places in the sequence $(w_k^*)$.
  The length of this sequence is $n^* = n + \#\{k > \tilde n : \tau_k = 0\}$.
  This sequence is then used to define the corrected trends $\tau_k^*$.

  \begin{equation*}
    \tau_k^* = \begin{cases}
      0, & k < \tilde n \\
      \tau(w_{k-1}^*, w_k^*), & k \ge \tilde n \text{ and } \tau(w_{k-1}^*, w_k^*) \ne 0 \\
      \tau_{k-1}^*, & \text{otherwise}
    \end{cases}
  \end{equation*}
  If a zero trend is found with the corrected windows as well, the last trend is carried forward.
  This ensures that every trend after the initial period is non-zero.

  The now non-zero trends are used to define trend changes, called \textit{reversals}, 
  of trends as $r_1 = 0$, $r_k = \sign(\tau_{k-1}^* - \tau_k^*), k > 1$.
  Let

  \begin{equation*}
    T^*(k) = \begin{cases}
      \emptyset, & r_k = 0 \\
      \{ l_{k-1} \}, & r_k = 1 \\
      \{ h_{k-1} \}, & r_k = -1 
    \end{cases}
  \end{equation*}
  If $T^*(k) \ne \emptyset$, then $T^*(k)$ is a singleton and 
  that element is called a \textit{reversal point}.

  This is used to define a new time series 
  $T^{(1)} = \bigcup_{k=1}^{n^*} T^*(k) \subset T$
  which contains only the points of T where a trend reversal happens.
  The reversal points can be used to determine the trend at 
  every point since the reversal points are the points at which the trend changes.
  The trend at a point $p_i$ is 
  
  \begin{equation*}
    \trend(p_i) = \begin{cases}
      0, & t_i < t_j \text{ for all } p_j \in T^{(1)} \\
      \tau^*(\tilde n) (-1)^{\#\{p_j \in T^{(1)} : t_j \le t_i\}}, & \text{otherwise}
    \end{cases}
  \end{equation*}
  i.e. the trend starts at 0 and switches sign at the reversal points.

  \subsubsection{Multiple iterations}

  The output of the algorithm, $T^{(1)}$,
  is then used as input for another iteration with 
  the same window size to get $T^{(2)} \subset T^{(1)}$.
  This is repeated until $\#T^{(m)} < s$, 
  at which point $T^{(m^*)} = \emptyset \; \forall m^* > m$ and 
  a decreasing sequence of sets $\mathcal{T}$ is given by
  $\mathcal{T} = (T, T^{(1)}, \dots, T^{(m)})$, where 
  $T \supset T^{(1)} \supset \cdots \supset T^{(m)}$.
  The \textit{order} of a reversal point $p$ is defined as 
  $\rho(p) = \max \{ k : p \in T^{(k)} \}$.
  A point $p$ that is not a reversal point, 
  i.e. $p \in T \backslash T^{(1)}$, 
  has reversal order 0, $\rho(p) = 0$.

  Since $\mathcal{T}$ is a decreasing sequence of sets, 
  the original time series $T$ can be augmented with 
  information about every reversal that has occurred thus far.
  Each point in the augmented time series $T_{tc}$ is 
  $(t_i,x_i,\rho_i) \in \mathbb{R}^2 \times \{ -m,\dots,m \}$, where 
  $\rho_i = \trend(p_i) \cdot \rho(p_i)$.

  If a point is a reversal of order $r$, 
  it is augmented with $r$ if it is 
  a local minimum and $-r$ if it is a local maximum. 
  If a point is not a reversal, it will be augmented with 0.

  Since $T^{(k)} = \{ p : \rho^*(p) \ge k \} $, 
  $T_{tc}$ contains all information necessary to recreate $\mathcal{T}$
  as well as information about the type of reversal.
  Hence, it also contains all information necessary to 
  determine the trend at every point of $T$.
\end{document}
