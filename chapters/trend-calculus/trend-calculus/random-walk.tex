% !TEX root = $uni/master-thesis/thesis/main.tex
\documentclass[../trend-calculus.tex]{subfiles}

\begin{document}
  Define a \textit{Trinomial Random Walk} as 
  the discrete time stochastic process $X_n$ given by 
  the sequence of partial sums of independent random variables $Y_i$, which 
  follow a given discrete distribution over $\{ -1, 0, 1 \}$, 
  i.e. $P(Y_i = k) = p_k^{(i)},\ k \in \{-1, 0, 1\}$.
  Then, $X_n = \sum_{i = 1}^n Y_i$.
  This results in a discrete, integer-valued time series that can, 
  in each step, either increase by 1, decrease by 1, or stay the same.

  For this showcase, $Y_i$ are taken to be identically distributed, 
  i.e. $p_k^{(i)} = p_k$ for all $i$.
  To further simplify the showcase, uniform probabilities are chosen, 
  $p_{-1} = p_0 = p_{1} = \nicefrac{1}{3}$.

  This yields a quite simple time series where it is 
  possible to precisely analyze the behavior of Trend Calculus.
  Since the higher order reversals are completely determined by the lower orders, 
  only the first order reversals are analyzed here.
  
  Whether a point is a trend reversal or not is 
  determined by the window the point belongs to, 
  the trend going into said window, 
  as well as the previous and following windows.
  To keep everything simple, the minimal window size of 2 is chosen. 

  To characterize the reversal points, the probability of 
  a certain sequence of reversals given 
  a sequence of points in the time series is calculated.
  Since all transition probabilities are equal, all sequences of points are equally likely.
  The sequence length must be a multiple of the window size, 
  here two windows (four points) are chosen as the sequence length.
  The number of different sequences of length $n$ is $3^n$, 
  so for a sequence of four points, as well as the previous and 
  following windows (eight points total), there are 
  $3^8 = 6561$ different sequences to consider.
  This is sufficiently small to study analytically.

  For each sequence of eight points $\{X_i\}_{i = 1}^8$, 
  the trend reversals are computed according to the Trend Calculus algorithm to 
  yield $\{R_i\}_{i=1}^8$.
  Since the first and last window are not of interest, 
  the output is $\{ (X_i, R_i) \}_{i = 3}^6$ for each of the possible sequences, 
  where $t$ is the trend going into the second window.
  The function $TC(\{X_i\}_{i = 1}^8) = \{ (X_i, R_i) \}_{i = 3}^6$ is not injective, and 
  since all sequences in the domain of $TC$ are equally likely, 
  the probability of a certain result sequence is calculated as follows.
  
  Let $\mathcal{D} = \{ \{X_i\}_{i=1}^8 \in \Zz^8 : X_0 = 0,\ |X_{i+1} - X_i| \le 1 \}$ be 
  the set of all possible Trinomial Random Walks of length 8, starting at 0, and 
  $I_S$ be the indicator function for $S$.
  Then 
  \[ 
    P( \{ (X_i, R_i) \}_{i=3}^6 ) = 
      \frac{1}{3^8} \sum_{\textbf{X} \in \mathcal{D}} 
        I_{ \{ (X_i, R_i) \}_{i=3}^6 } (TC(\X)) 
  \]

  This is used to calculate the probability for
  having a certain number of reversals within the inner two windows.
  The results of this calculation are presented in Table \ref{tab:tc-showcase}.
  The trinomial random walk is studied to contrast the trend reversals with those of
  a financial time series from real data.
  Figure \ref{fig:trendcalculus} shows a few realizations of
  the trinomial random walk time series as well as the trends
  and reversal orders of oil and gold prices over the same time series length.

  \begin{table}
    \centering
    % | no. revs | prob |
    \begin{tabular}{|c|c|}
      \hline
      Number of reversals & probability (\%) \\
      \hline
      0 & 32.7 \\
      1 & 47.7 \\
      2 & 18.5 \\
      3 & 1.1 \\
      \hline
    \end{tabular}
    \caption{
      Probabilities for different number of reversals in a
      trinomial random walk of length 4 with uniform transition probabilities.
    }
    \label{tab:tc-showcase}
  \end{table}

\end{document}