% !TEX root = $uni/master-thesis/thesis/main.tex
\documentclass[../main.tex]{subfiles}

\begin{document}
  Estimating histograms is one of the most natural approaches in density estimation.
  The main drawback of histograms is the computational resources needed when 
  the number of dimensions and/or the sample size is very large.

  A partial solution to this problem is to make the computation scalable,
  distributing the workload across several workers.
  In the best case scenario, such distributed computing
  decreases the computation time linearly in the number of workers.
  
  The \verb|mrs2| library for C++ \cite{mrs2,mrp-raaz-harlow-tucker}
  contains efficient implementations for non-distributed estimation of histograms as MRPs.
  A distributed version implemented in Scala and using Apache Spark for distribution
  is given in \cite{scala-density-tree}.
  In this work, the randomized algorithms for scalable density estimation
  based on minimum distance in \cite{srp-mde-raaz-teng}
  are made efficient with sparse MRPs.

  \section{Histogram estimation through merging}
    \subfile{scalable-hists/merging.tex}

  \section{Minimum Distance Estimation}
    \subfile{scalable-hists/mde.tex}

  \section{Results}
    \subfile{scalable-hists/results.tex}

  \section{Discussion}
    \subfile{scalable-hists/discussion.tex}
\end{document}