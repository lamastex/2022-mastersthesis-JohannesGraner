% !TEX root = $uni/master-thesis/thesis/main.tex
\documentclass[../scalable-gopt.tex]{subfiles}

\begin{document}
  Global optimization refers to finding the 
  global minimum or maximum value of a function 
  $f: \Rz^d \to \Rz$ in a specified domain $\X \in \Rz^d$.
  Without loss of generality, the minimum is sought, 
  since the maximum is found by minimizing $-f$.
  In order to use the advantages of interval arithmetic, 
  some assumptions on $f$ are necessary.
  The most basic assumptions are i) $f$ has an inclusion isotonic interval extension $F$,
  ii) $\X$ is compact, and iii) $f$ is well-defined everywhere in $\X$.

  The domain 
  $\X = \left( \X \ind 1, \dots, \X \ind n \right) = 
  \left( [\ul X \ind 1, \ol X \ind 1], \dots, [\ul X \ind n, \ol X \ind n] \right)$ is
  then partitioned component-wise as 
  $\X \ind j = \bigcup_{i=1}^{b} [x_i \ind j, x_{i+1} \ind j]$, 
  where $x_1 \ind j= \ul X \ind j$, $x_{b+1} \ind j = \ol X \ind j$, 
  and $i < k \implies x_i \ind j< x_k \ind j$.
  In other words, $\X$ is divided into $b$ thick intervals 
  $\x_i = (\x_i \ind 1, \dots, \x_i \ind n), i = 1,\dots,b$ 
  that overlap only at the boundaries.
  The general approach is to test if a global minimum can occur in one of the intervals in the partition, 
  and discard the interval if such is not the case.
  The remaining intervals in the partition are then bisected along the axis of maximum diameter and
  the process is repeated until all remaining intervals have
  (maximum) relative diameter no larger than a tolerance $\epsilon$.
  Let $\F_i = F(\x_i)$.
  Four such tests are used, along with a verification process,
  all of which are explained in detail in \cite{toolbox-hammer}.

  \subsection{Midpoint cut-off test}
    \subfile{global-optimization/midpoint-cutoff.tex}

  \subsection{Monotonicity test}
    \subfile{global-optimization/monotonicity.tex}

  \subsection{Concavity test}
    \subfile{global-optimization/concavity.tex}

  \subsection{Interval Newton test}
    \subfile{global-optimization/newton.tex}

  \subsection{Verification}
    \subfile{global-optimization/verification.tex}

\end{document}