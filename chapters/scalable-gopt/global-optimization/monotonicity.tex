% !TEX root = $uni/master-thesis/thesis/main.tex
\documentclass[../global-optimization.tex]{subfiles}

\begin{document}
  The monotonicity test requires that $f$ is 
  continuously differentiable everywhere in $\X$, and 
  determines if $f$ is strictly monotone over a whole interval.
  A strictly monotone function over a compact interval 
  must have its minimum value on the boundary of said interval.
  If $\x$ is such an interval contained in the interior of $\X$, 
  then $\x$ can be discarded since there is an adjacent box that is 
  guaranteed to contain a point with lower function value than any point in $\x$.
  If $\x$ intersects the boundary of $\X$, 
  then $\x$ can only be discarded if this intersection 
  does not contain $\ul F(\x)$, which can be determined from the monotonicity.

  The test is as follows, given $\x$, $\X$ and $\nabla F(\x)$.

  \begin{itemize}
    \item For $i = 1,\dots,n$:
    \begin{enumerate}
      \item If $0 \in \nabla F(\x)_i$, go to next index.
      \item Otherwise, $f$ is monotone in at least one component over $\x$ and 
      hence $\x$ can be removed unless $\x \cap \X \neq \emptyset$, 
      in which case the following is checked.
      \begin{itemize}
        \item If $\min \nabla F(\x)_i > 0$ and $\ul X \ind i = \ul x \ind i$, then modify $\x$ to be the thin interval $[\ul x \ind i, \ul x \ind i]$.
        \item Else, if $\max \nabla F(\x)_i < 0$ and $\ol X \ind i = \ol x \ind i$, then modify $\x$ to be the thin interval $[\ol x \ind i, \ol x \ind i]$.
        \item Else, remove $\x$ from the search and stop the iteration.
      \end{itemize}
    \end{enumerate}
  \end{itemize}
\end{document}