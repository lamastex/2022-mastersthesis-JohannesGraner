% !TEX root = $uni/master-thesis/thesis/main.tex
\documentclass[../global-optimization.tex]{subfiles}

\begin{document}
  When there are no remaining interval with maximal relative diameter greater than $\epsilon$, 
  the candidate intervals are checked to see if a minimum can be verified in them.
  Note that this step is only possible when 
  $f$ is twice continuously differentiable, 
  and is done by checking two conditions.
  First, if $N'_{GS}(\x) \subseteq \Iint(\x)$, 
  then both existence and uniqueness of a stationary point in $\x$ is verified.
  Second, the positive definiteness of $\nabla^2 F(\x)$ is 
  checked to verify that the stationary point is a minimum.
  This is done by the following theorem, Theorem 14.1 in \cite{toolbox-hammer}.

  \begin{theorem}
    Let $\H \in \Iz \Rz^{n \times n}$ and $S$ be defined by $\S = I - \frac{1}{\kappa} \H$, with \linebreak $\dmrel \H \le \kappa \in \Rz$.
    If $\S$ satisfies
    \begin{equation*}
      \S \cdot \z \subseteq \Iint (\z)
    \end{equation*}
    for an interval vector $\z \in \Iz \Rz^n$, then $\rho(B) < 1$ for all $B \in \S$, and all symmetric matrices $A \in \H$ are positive definite.
  \end{theorem}
  where $\H = \nabla^2 F(\x)$.

  Note that this can only verify that a unique minimizer is found in the interval,
  there are no guarantees that a given candidate interval contains 
  a global minimizer except in the case where there is only one candidate interval.
  If an interval cannot be verified to contain a minimizer, 
  that interval should still be kept in the list of candidates since, 
  for example, the global minimum could be a continuum of points in that interval.

\end{document}