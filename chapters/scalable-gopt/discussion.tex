% !TEX root = $uni/master-thesis/thesis/main.tex
\documentclass[../scalable-gopt.tex]{subfiles}

\begin{document}
  In both cases, running the original global optimization algorithm (i.e. without distributed computing) 
  proved to be both faster and able to handle higher dimensions than the scalable extension.

  The issue is that the global optimization algorithm is 
  extremely slow on certain intervals, 
  such as those containing the origin, but not the global minimum, 
  in case of the Rosenbrock function.
  This creates a situation where the worker assigned to 
  the problematic interval takes much longer to finish than 
  any of the other workers, leading to all but one worker being 
  idle for the majority of each iteration.
  Addressing this issue would require either improving the original algorithm or 
  a way to interrupt computations on other workers once 
  a better candidate interval is found.
  
  Unfortunately, improving the original algorithm is outside the scope of this thesis, 
  and mid-computation communication between workers is not possible in native Apache Spark, 
  hence another framework (e.g. MPI) would be required.
  There are efforts to do mid-computation communication in Apache Spark, 
  for example in the framework Maggy \cite{maggy}.
  However, Apache Spark was chosen due to the ease of setting up the cluster and distributing the workload.
  Using a more complex framework is also outside the scope of the thesis, 
  leading to the conclusion that while it should be possible to implement a 
  scalable global optimization routine as Algorithm \ref{alg:gopt-scalable}, 
  an efficient implementation is not presented here.

\end{document}