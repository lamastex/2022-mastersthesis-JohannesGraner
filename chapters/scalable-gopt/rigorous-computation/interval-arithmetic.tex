% !TEX root = $uni/master-thesis/thesis/main.tex
\documentclass[../rigorous-computation.tex]{subfile}

\begin{document}
  Both measurement errors and round-off errors can be carried forward properly through computations using interval arithmetic.

  An interval $\x = [\ul x, \ol x] \subset \Rz$ can be used to encapsulate the error by using the lower and upper bounds for the number or measurement as the lower and upper bound of the interval, respectively.
  The set of all real intervals is $\Iz \Rz = \{[\ul x, \ol x] : \ul x \le \ol x,\; \ul x, \ol x \in \Rz \}$.
  The diameter of $\x$ is $d(\x) = \ol x - \ul x$ and the midpoint is $\Imid(\x) = \frac{\ul x + \ol x}{2}$.
  The largest absolute value of $\x$ is 
  $|\x| = \max \{ |x| : x \in \x \} = \max \{ |\ul x|, |\ol x| \}$ and 
  the smallest absolute value is \linebreak 
  $\left< \x \right> = \min \{ |x| \in \x \} = \begin{cases} 0,& 0 \in \x \\ \min \{ |\ul x|, |\ol x| \},& 0 \notin \x \end{cases}$.
  The absolute value of $\x$ is $|\x|_{[]} = [\left< \x \right>, |\x|]$.
  A useful measure is the relative diameter of an interval, denoted $d_{rel}(\x)$.
  If $0 \notin \x$, then $\drel(\x) = \frac{d(\x)}{\left< \x \right>}$, 
  otherwise $\drel(\x) = d(\x)$.
  An interval with $\ul x = \ol x$ is called \textit{thin}.
  An interval that is not thin, i.e. $\ul x < \ol x$, is called \textit{thick}.

  Intervals are not restricted to the real line, 
  in particular they can be extended to interval vectors in $\Iz\Rz^d$ by 
  letting each component be an interval.
  The diameter, midpoint, etc. are then vectors in $\Rz^d$ and 
  the maximal diameter of $\x \in \Iz \Rz^d$ is defined by $\dm(\x) = \max_i d(\x_i)$.
  The maximal relative diameter is similarly defined as $\dmrel(\x) = \max_i \drel(\x_i)$.

  A metric $g$ can be defined on $\Iz \Rz$ as 
  $g(\x,\y) = max\{|\ul x - \ul y|, |\ol x - \ol y|\}$, 
  yielding that $\Iz \Rz$ is a complete metric space under $g$.
  The sequence of intervals $\{\x^{(i)}\}$ converging to $\x$ is 
  equivalent to $\ul x^{(i)} \to \ul x$ and $\ol x^{(i)} \to \ol x$.
  This can also be extended to $\Iz \Rz^d$ and 
  hence it is reasonable to consider continuous and differentiable functions 
  $f : \Iz \Rz^d \to \Iz \Rz^k$ with continuity and differentiability 
  defined as usual with the metric $g$.

  With $\Iz \Rz$ as a metric space, an arithmetic is defined by 
  the operations $\{+,-,\cdot,/\}$.
  Let $\circ \in \{+,-,\cdot,/\}$ and $\x, \y \in \Iz \Rz$.
  Assume $0 \notin \y$ when $\circ = /$.
  Then, $\x \circ \y$ is defined by 
  $\x \circ \y = \{ x \circ y : x \in \x, y \in \y \}$.
  Using this definition of the arithmetic operations makes them \textit{inclusion isotonic}; 
  if $\vv \subseteq \x$ and $\w \subseteq \y$, then $\vv \circ \w \subseteq \x \circ \y$.
  This is due to $\x$ and $\y$ being simply connected compact intervals, 
  and $\circ$ a continous operation.
  With these properties, $\x \circ \y$ contains the minimum, maximum and 
  all values in between of $x \circ y$ where $x \in \x, y \in \y$, 
  i.e. $\x \circ \y = [\min\{ x \circ y \}, \max\{ x \circ y \}]$.

  The binary operations of $+$ and $\cdot$ have identity elements, the thin intervals $[0,0]$ and $[1,1]$, respectively.
  However, neither addition nor subtraction have inverses except when $\x$ is thin, 
  due to $[0,0] \subsetneq \x - \x$ and $[1,1] \subsetneq \x / \x$ whenever $\ul x < \ol x$.
  Both $+$ and $\cdot$ are associative and commutative, but only subdistributive, i.e. $\x \cdot (\y + \z) \subseteq (\x + \y) \cdot (\x + \z)$.

  If $f : \Rz^d \to \Rz$ is a function, 
  then the image of $f$ over an interval vector $\x$ is 
  $f(\x) = \{ f(x) : x \in \x \}$. 
  Further, if $f$ is a combination of elementary functions (sin, cos, exp, etc.), 
  arithmetic operations, and constants, 
  then $f$ is inclusion isotonic and can be extended to its 
  \textit{natural interval extension} $F : \Iz \Rz^d \to \Iz \Rz$ by 
  replacing all components of $f$ with their interval counterparts.
  Since $F$ is inclusion isotonic, for each $x \in \x$, $f(x) \in F(\x)$.
  Since this implication is only one way, 
  in general $F(\x) \not\subseteq f(\x)$, 
  and hence $F(\x)$ is generally an overestimation of $f(\x)$.
  However, as $\dm(\x)$ decreases to 0, 
  the difference between $F(\x)$ and $f(\x)$, $g(F(\x),f(\x))$ 
  decreases linearly to 0 as well.
  In other words, by partitioning $\x$ as $\{\x^{(1)},\dots,\x^{(m)}\}$, 
  a better approximation of $f(\x)$ is achieved by $\bigcup_i F(\x^{(i)})$.

  An even better approximation is available when $f$ is 
  differentiable everywhere in $\x$.
  The \textit{centered form} of $f$ is based on the mean value theorem.

  \begin{align*}
    f(x) = & f(c) + \nabla f(b) \cdot (x - c) \in f(c) + \nabla f(\x) \cdot (x - c) \subseteq \\
    \subseteq & f(c) + \nabla F(\x) \cdot (\x - c) = F_c(\x)
  \end{align*}

  where $b,c,x \in \x$ with $c < b < x$ and real numbers are 
  treated as thin intervals where necessary.
  $g(F_c(\x),f(\x))$ decreases to 0 quadratically as $\dm (\x) \to 0$.

  Another way to improve the approximation $F(\x)$ is to 
  simplify the expression for $f(x)$ in order to 
  minimize the number of occurances of $x$, and 
  use thin intervals for constants and parameters wherever possible.

  For an in-depth treatment of interval arithmetic, see e.g. \cite{toolbox-hammer}.
  

  
\end{document}