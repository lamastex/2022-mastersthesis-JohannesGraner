% !TEX root = $uni/master-thesis/thesis/main.tex
\documentclass[../main.tex]{subfiles}

\begin{document}
  Time series often feature regions where the tendency to 
  increase or decrease is somewhat constant.
  These regions can vary dramatically in scale while behaving quite similarly.
  Such behaviors are called \textit{trends} and are of much interest in many disciplines, 
  among them analysis of financial time series.
  The time series tends to increase when the trend is \textit{up}, 
  and decrease when the trend is \textit{down}.

  While such trends can be easy to spot for a human, 
  it is not trivial how a trend should be defined precisely for use in automated systems,
  which are necessary when the time series is so long that a human cannot find
  all short-term trends in a reasonable amount of time.
  One possible definition of a trend which lends itself nicely to 
  fast computation is the one used in Trend Calculus \cite{trendcalculus,morgan-amend-mastering-spark}.

  \section{Trend Calculus}
    \subfile{trend-calculus/trend-calculus.tex}

  \section{Discussion}
    \subfile{trend-calculus/discussion.tex}
\end{document}