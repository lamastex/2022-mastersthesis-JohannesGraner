% !TEX root = $uni/master-thesis/thesis/main.tex
\documentclass[../mapped-regular-pavings.tex]{subfiles}

\begin{document}
  
  Let $s \in \Sz_{0:\infty}$ be an RP with root box $\x_\rho$ and let $\Az$ be a non-empty set.
  By augmenting each leaf node $l \in \Lz(s)$ with an element in $\Az$, 
  a piece-wise constant function $f: \Lz(s) \to \Az$ is defined.
  This function can be extended to a piece-wise function $f_x: \x_\rho \to \Az$ by 
  first mapping $x \in \x_\rho$ to $l(x) \in \Lz$, and
  then applying $f$, i.e.~$f_x = f \circ l$.
  This is well-defined due to $s$ being a partition of $\x_\rho$, so 
  $x$ can only belong to a single leaf box.
  An RP $s$ together with the set $\Az$ and 
  piece-wise constant function $f$ is referred to as an $\Az$-MRP.

  Since RPs are stored as sorted lists of leaves, 
  it is natural to store MRPs in a very similar way.
  Each leaf $l$ is augmented with $f(l)$ and stored in a list 
  $[(l_1, f(l_1)), (l_2, f(l_2)), \dots, (l_m, f(l_m))]$.
  If $s$ is an $\Az$-MRP, let $\Lz(s)$ be the underlying RP as an ordered set of leaves.
  As described in \cite{mrp-raaz-harlow-tucker}, it is possible to 
  define operations on MRPs with the same root box $\x_\rho$.

  \subsection{Sparse MRPs}
    \label{sec:sparse}
    \subfile{mrp/sparse.tex}

  \subsection{Operations on MRPs}
    \subfile{mrp/operate.tex}

\end{document}