% !TEX root = $uni/master-thesis/thesis/main.tex
\documentclass[../mrp.tex]{subfiles}

\begin{document}
  Depending on what the MRP represents, the set $\Az$ may 
  contain an element that is natural to consider as a base, or neutral element.
  For example, let $\Az = \{ \True, \False \}$ and the MRP represents a 
  discretization of some geometric object where boxes that
  intersect the object are given the $\True$ value and all other boxes have the $\False$ value.
  Then the value $\False$ could be considered as the neutral element, 
  representing the lack of intersection with the object.

  Similarly, let $\Az = \Zz_{\ge 0}$ and the MRP represents 
  the number of data points that fall into each box,
  then the value $0$ would be the base element, representing the lack of any data.
  If such a neutral element $e \in \Az$ exists, 
  the memory footprint of the MRP can be reduced significantly by 
  only storing the leaves associated with a value other than $e$.
  An MRP in this compressed format is called a \textit{sparse} MRP since 
  the set of leaves stored is sparse in the the full set of leaves.
  An MRP that is not sparse is called \textit{dense}.

  Let $s_D = \{ (l_1, a_1),\dots,(l_m, a_m) \}$ be 
  a dense $\Az$-MRP with a neutral element $e \in \Az$.
  The corresponding sparse $\Az$-MRP is then $s_S = \{ (l_i, a_i) \in s_D : a_i \neq e \}$.
  An example of a sparse MRP is given in Figure \ref{fig:sparse-mrp}.

  \subfile{sparse/sparse-graphic.tex}

  In order to use sparse MRPs effectively, 
  algorithms for RP or MRP operations have to work even when all leaves are not present.
  Algorithms \ref{alg:rpunion} and \ref{alg:mrpoperate} are both written with sparse MRPs in mind,
  and work for both sparse and dense trees.

  However, the result of $\RPUnion(s \ind 1, s \ind 2)$ is often much less sparse than $s \ind 1$ or $s \ind 2$.
  This is because when one RP contains a node that is an ancestor to a node in the other tree, 
  all missing nodes in that subtree have to be included in the union.
  For example, if $s \ind 1 = \{ \rho_L \}$, and 
  $s \ind 2 = \{ \rho_{LLL} \}$ are sparse Boolean MRPs with $e = \False$, then 
  $\RPUnion(s \ind 1, s \ind 2) = \{ \rho_{LLL}, \rho_{LLR}, \rho_{LR} \}$ 
  in order to retain the information that 
  both $\rho_L$ and $\rho_{LLL}$ were part of the original trees.
  This is illustrated in Figure \ref{fig:sparse-union}.
  In this case, the union could be represented with the second operand
  by combining leaves associated with the same value.
  This is implemented as $\mathrm{MinimiseLeaves}$ in \cite{mrp-raaz-harlow-tucker}.
  One could implement that algorithm for this representation of MRPs, but is not done here.
  Figure \ref{fig:mrpoperate} shows a scenario where combining leaves is not possible.

  \subfile{sparse/sparse-union-graphic.tex}
  
\end{document}