% !TEX root = $uni/master-thesis/thesis/main.tex
\documentclass[../mrp.tex]{subfiles}

\begin{document}
  
  The fundamental operation on $\Az$-MRPs is described in Algorithm \ref{alg:mrpoperate}.
  This operation takes two $\Az$-MRPs with the same root box and
  applies a function $\star: \Az \times \Az \to \Az$ to each pair of overlapping leaves.
  MRPOperate is introduced as Algorithm 4 in \cite{mrp-raaz-harlow-tucker} and 
  is adapted for bitstring representation and sparse MRPs in 
  Algorithm \ref{alg:mrpoperate} below.

  To accomodate sparse MRPs, the neutral element $e \in \Az$ must also be supplied to the algorithm.
  If both MRPs are dense, then the neutral element argument is not used.
  This ensures that the same implementation works for both sparse $\Az$-MRPs (as long as there is a neutral element in $\Az$), and for dense $\Bz$-MRPs where $\Bz$ does not necessarily contain a neutral element.
  An example of $\MRPOperate(s \ind 1, s \ind 2, +, 0)$ with $s \ind 1, s \ind 2$ being $\Zz$-MRPs is shown in Figure \ref{fig:mrpoperate}.

  \begin{algorithm}
    \caption{MRPOperate($s \ind 1, s \ind 2, \star, e$)}
    \label{alg:mrpoperate}
    \KwIn{
      $s \ind 1, s \ind 2$, $\Az$-MRPs with the same root box, 
      
      $\star: \Az^2 \to \Az$, 
      
      $e \in \Az$, the neutral element (if $s \ind 1$ and $s \ind 2$ are both dense, then $e$ is not used).}
    \KwOut{$s$, the $\Az$-MRP with the operation applied to overlapping leaves.}

    $s_L \gets \RPUnion(\Lz(s \ind 1), \Lz(s \ind 2))$

    $s \gets []$
    
    $L_1 \gets s \ind 1$,
    $L_2 \gets s \ind 2$

    \For{$l \in s_L$}{
      \eIf{$\exists (l_1, v_1) \in L_1$ \normalfont{such that} $t(l, d(l_1)) = l$}{
        $v_1' \gets v_1$
        // matching node's or ancestor's value is passed down

        \If{$l_1 = l$}{
          $i \gets \mathrm{index}((l_1, v_1), L_1)$ // index starting at 1

          $L_1 \gets L_1$ with the first $i$ elements removed
        }
      }{
        $v_1' \gets e$
      }

      \eIf{$\exists (l_2, v_2) \in L_2$ \normalfont{such that} $t(l, d(l_2)) = l$}{
        $v_2' \gets v_2$
        // matching node's or ancestor's value is passed down

        \If{$l_2 = l$}{
          $i \gets \mathrm{index}((l_2, v_2), L_2)$ // index starting at 1

          $L_2 \gets L_2$ with the first $i$ elements removed
        }
      }{
        $v_2' \gets e$
      }

      $s \gets (l, v_1' \star v_2') :: s$
    }

  \Return{$s$}

  \end{algorithm}

  \subfile{operate/operate-graphic.tex}

\end{document}
