% !TEX root = $uni/master-thesis/thesis/main.tex
\documentclass[../mapped-regular-pavings.tex]{subfiles}

\begin{document}
  
  Let $x_{[n]} = (x_1, \dots, x_n)$ be a data set with $x_i \in \Rz^d$,
  and let $\B \subset \Rz^d$ be an interval vector (box) such that $x_i \in \B$ for all $i$.
  Note that $\x_{[n]}$ is not necessarily a set,
  multiple occurrences of the same data point are allowed.

  A histogram of $x_{[n]}$ is then a partition of $\B$ into $\b_1, \dots, \b_m$ such that
  $\b_i \cap \b_j = \emptyset$ for all $i \neq j$ and
  $\bigcup_{i = 1}^m \b_i = \B$,
  together with the counts $\#x_1, \dots, \#x_m$ of the number of data points in each $\b_i$,
  i.e. $\#x_i = \sum_{j=1}^n \mathbbm{1}(x_j \in \b_i)$.

  Restricting the possible partitions to regular pavings of $\B$,
  a natural way to represent histograms is $\Zz$-MRPs.
  If the dimensionality $d$ of the data is small,
  a dense $\Zz$-MRP can be enough.
  However, when $d$ is large the histogram suffers from a curse of dimensionality since
  the RP requires a much larger number of splits,
  leading to a very large number of sub-boxes with no data ($\#x_i = 0$).
  A solution to this problem is to use sparse $\Zz$-MRPs with neutral element 0.

  Strategies for partitioning $\B$ into an RP based on stopping criteria 
  to obtain minimum distance histogram esimates are 
  presented in \cite{srp-mde-raaz-teng}, and 
  a scalable implementation using Apache Spark and sparse bitstring RPs is 
  given in \cite{scala-density-tree}.
  Here, an alternative scalable implementation is presented,
  one that is based on merging leaves of a sparse histogram instead
  of splitting leaves of a dense histogram.

  Assume $x_{[n]}$ is an independent and identically distributed sample of size $n$ from 
  an underlying unknown density $f \in \mathcal{L}_1$, i.e. $f$ is any density.
  A count based histogram of $x_{[n]}$ with 
  $\Zz$-MRP $s$ can be used to estimate the density $f$ on $\B$ by
  letting 
  $f_{n, s}(x) = \frac{\#\x_{\rho \vee}}{n \, \vol (\x_{\rho \vee)})}$ if 
  $x \in \x_{\rho \vee}$ and $\x_{\rho \vee}$ is a leaf box.

  The density estimate $f_{n, s}$, or $f_n$ for notational convenience, 
  can be seen as a sparse $\Rz$-MRP with neutral element $0$.
  This is also a type of histogram, 
  using frequency density of boxes instead of frequency count.
  When there is a need to distinguish between the two types of histograms,
  they are referred to as count-based or density-based, respectively.

  Since a density-based histogram is a density estimate,
  operations that are usually performed on densities can
  be performed on density-based histograms as well.
  Two of the most important operations on densities are
  marginalization and finding conditional densities, 
  as together they allow for regression based on 
  conditional densities of the joint density estimate.

  \subsection{Marginalization}
    \subfile{histograms/marginalization.tex}

  \subsection{Conditional density}
    \subfile{histograms/slice.tex}

\end{document}