% !TEX root = $uni/master-thesis/thesis/main.tex
\documentclass[../mapped-regular-pavings.tex]{subfiles}

\begin{document}
  
  Consider an interval vector (or box) $\x \in \Iz \Rz^d$.
  The first coordinate of maximum width is denoted $\iota$.
  By splitting $\x$ perpendicularly at the mid-point along the $\iota$-th coordinate, 
  the child boxes $\x_L$ and $\x_R$ are given as

  \begin{align*}
    \x_L &= \x_1 \times \dots \times [\ul x_\iota, \fct{mid} (\x_\iota) ) \times \x_{\iota + 1} \times \dots \times \x_d, \\
    \x_R &= \x_1 \times \dots \times [\fct{mid} (\x_\iota), \ol x_\iota ] \times \x_{\iota + 1} \times \dots \times \x_d.
  \end{align*}

  This type of bisection is called \textit{regular}.
  Since the split interval in $\x_L$ is half-open, 
  the intersection between $\x_L$ and $\x_R$ is empty.
  Hence, $\{ \x_L, \x_R \}$ constitutes a partition of $\x$.
  A recursive sequence of regular bisections starting from the root box $\x_\rho$ is 
  called a \textit{regular paving} or $n$-tree \cite{scala-density-tree}.
  A regular paving of $\x_\rho$ can be represented by a binary tree, 
  where each node in the tree corresponds to a sub-box of $\x_\rho$.
  In such a tree, each node has either zero or two children.
  The set of leaf nodes, denoted $\Lz$, constitutes 
  a partition of the root box $\x_\rho$ since each box corresponding to 
  a leaf node is an element of the partition of the box corresponding to the parent of all the leaf nodes.

  Figure \ref{fig:tree-graphic} gives the intuition for this, 
  the label of each leaf node corresponding to a box in the partition.
  Note that the volume of a box in the partition is determined by 
  the depth of the corresponding node in the tree.
  Likewise, the position of the box is determined by 
  the position of the node compared to all other possible nodes at that depth.
  For example, in Figure \ref{fig:tree-graphic} the node $\rho LL$ is the leftmost node at depth 2,
  hence the corresponding box $\x_{\rho LL}$ is in the bottom left corner since 
  this is the box given by taking the lower interval in each split.
  For $x \in \x_\rho$, let $l(x) \in \Lz$ be the leaf node 
  corresponding to the finest partitioned box that $x$ belongs to.

  \subfile{regular-pavings/tree-graphic.tex}

  Given a root box $\x_\rho$, let $\Sz_k$ be 
  the set of all regular pavings with $k$ splits starting from $\x_\rho$.
  Since $|\Lz(s)| = 1$ if $s \in \Sz_0$ and each split adds exactly one leaf, 
  $m(s) = |\Lz(s)| = k + 1$ if $s \in \Sz_k$.
  Let $\Sz_{i:l} = \bigcup_{k = i}^l \Sz_k$ be the set of 
  regular pavings with between $i$ and $l$ splits, inclusive.
  The set of all regular pavings is denoted $\Sz_{0:\infty} = \lim_{l \to \infty} \Sz_{0:l}$.

  \subsection{Memory-efficient representation}
    \label{sec:leaves-only-rp}
    \subfile{regular-pavings/leaves-only.tex}

  \subsection{Union of regular pavings}
    \subfile{regular-pavings/union.tex}

\end{document}