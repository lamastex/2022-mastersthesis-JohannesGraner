% !TEX root = $uni/master-thesis/thesis/main.tex
\documentclass[../regular-pavings.tex]{subfiles}

\begin{document}
  
  Every node in an RP $s \in \rps$ has zero or two children, and 
  a node has zero children if and only if it is a leaf node.
  Each leaf node can also be uniquely determined by its sequence of splits from the root node, 
  which contains -- for each split -- the information of whether the node belongs to 
  the left or right child node in the split.
  In other words, a leaf node can be represented as a sequence of bits (0 for left, 1 for right).
  Such a sequence is called a \textit{bitstring}.
  Using this representation, the RP is uniquely determined by the set of leaf node bitstrings.
  Considering that the leaves correspond to a partition of the root box, 
  these bitstrings act as labels for the boxes in the partition.

  The depth of a node $l$, denoted as $d(l)$, is given by the length of $l$'s bitstring.
  Denote the ancestor of $l$ at depth $d$ by $t(l, d)$ and 
  note that $t(l, d)$ is given by the first $d$ bits in $l$'s bitstring.
  Hence $t(l, d)$ is the truncation of $l$'s bitstring to its first $d$ bits.
  To make $t(l, d)$ defined for all $d \in \Zz_{>0}$, let $t(l, d) = t(l, d(l)) = l$ if $d \ge d(l)$.

  This allows for efficient storage of the RP, 
  since there is no need to keep track of the internal nodes explicitly.
  Representing the tree in this way differs from \cite{mrp-raaz-harlow-tucker} 
  where every node is present.
  Using the bitstring representation necessitates changes to 
  the algorithms presented in \cite{mrp-raaz-harlow-tucker}.
  
  \subsubsection{Ordering}
  Although the memory footprint of the RP is not affected by the ordering of leaves,
  several operations can be made more efficient by requiring a certain order of the leaves.

  A left-to-right ordering is given by $l_1 < l_2$ if and only if $t(l_1, d(l_2))$ is 
  to the left of $l_2$, or $t(l_2, d(l_1))$ is to the right of $l_1$.
  In other words, to determine if $l_1 < l_2$, the two nodes are truncated to the same depth 
  (the minimum of $d(l_1)$ and $d(l_2)$) where they can be directly compared.

  For example, in Figure \ref{fig:tree-graphic}, $\rho LRL < \rho RL$ since $t(\rho LRL, d(\rho RL)) =$ \linebreak $= t(\rho LRL, 2) = \rho LR$ which is clearly left of $\rho RL$.

\end{document}