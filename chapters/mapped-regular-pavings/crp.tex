% !TEX root = $uni/master-thesis/thesis/main.tex
\documentclass[../mapped-regular-pavings.tex]{subfiles}

\begin{document}
  There are situations where
  several different histograms are needed at the same time,
  such as in density estimation where histograms are 
  compared to determine which is the best estimate of
  the underlying density.
  For this reason, a data structure that contains several histograms can be advantageous.

  One such data structure is presented in 
  \cite{srp-mde-raaz-teng} as the \textit{Collated Regular Paving} (CRP).
  A CRP is an RP where each leaf is associated with more than one value.
  In \cite{srp-mde-raaz-teng}, the values associated with each leaf is
  the density of each histogram in the box that the leaf represents,
  as well as the empirical density of validation data in the box.
  Adding an additional histogram to the collection is referred to as \textit{collating},
  and the same term is used when a CRP is created from two non-collated histograms.
  To collate two MRPs, they must have the same root box,
  so that the leaf labels refer to the same sub-boxes.

  Note that any two $\Az-$MRPs with the same root box can be collated,
  this data structure is not unique to histograms.
  The only implementation presented here is for density-based histograms,
  but the implementation details for other MRPs would be very similar.

  A procedure for collation exists in \verb|mrs2| \cite{mrs2},
  and the corresponding procedure for (sparse) bitstring representations
  of MRPs is presented here in Algorithm \ref{alg:crp}.
  Only the collation of two CRPs is presented since
  an MRP can easily be transformed into a CRP with only one underlying MRP.
  In this implementation, a CRP is an MRP where each leaf is associated
  with a map from $k \in \Kz$, where $\Kz$ is a set of labels (keys) for the histograms,
  to the density estimated by histogram $f_k$ in the leaf's box.
  In other words, a CRP is a $\{ \kappa : \Kz \to \Rz_{\ge 0} \}$-MRP.
  Every leaf has the same key set $\Kz$, so some leaves may 
  have estimated densities of 0 for one or more histograms.
  This generally makes the CRP less sparse than the underlying histograms.
  If $s$ is a CRP, let $K(s)$ denote the set of keys in $s$.

  \begin{algorithm}
    \caption{Collation of MRPs}
    \label{alg:crp}
    \SetKwFunction{FCollate}{$\mathrm{collatorOp}$}
    \SetKwProg{Fn}{Function}{:}{}
    \KwIn{
      $s_1, s_2$, CRPs of density-based histograms with non-overlapping key sets
    }
    \KwOut{
      $s$, the collation of $s_1$ and $s_2$
    }
    $\Kz_1 \gets K(s_1)$;
    $\Kz_2 \gets K(s_2)$

    \Fn{\FCollate{$m_1$, $m_2$}}{
      // $m_1$ and $m_2$ are maps from keys to densities,
      represented as sets $\{ k \mapsto v \}$

      \If(// leaf does not exist in $s_1$){$m_1 = \emptyset$}{
        \Return{
          $m_2 \cup \{ k \mapsto 0 : k \in \Kz_1 \}$
        }
      } \ElseIf(// leaf does not exist in $s_2$){$m_2 = \emptyset$}{
        \Return{
          $m_1 \cup \{ k \mapsto 0 : k \in \Kz_2 \}$
        }
      } \Else(// leaf exists in both $s_1$ and $s_2$){
        \Return{
          $m_1 \cup m_2$
        }
      }
    }

    \Return{$\MRPOperate(s_1, s_2, \fct{collatorOp}, \emptyset)$}
  \end{algorithm}
\end{document}